\documentclass[a4paper,12pt]{article}
\usepackage[slovene]{babel}
\usepackage[utf8]{inputenc}
\usepackage[T1]{fontenc}
\usepackage{amsmath, amssymb}
\usepackage{graphicx}
\usepackage{geometry}
\usepackage{amsthm}

\theoremstyle{definition}
\newtheorem{definition}{Definicija}

\theoremstyle{remark}
\newtheorem*{remark}{Opomba}

\geometry{margin=2.5cm}
\setlength{\parskip}{0.8em}
\setlength{\parindent}{0pt}

\begin{document}

    \begin{center}
        \Large\textbf{Število podpoti pri grafih z danim ciklomatskim številom}\\[4pt]
        \large \textit{Poročilo projektne naloge}\\
        \large \textit{Skupina 18: Maja Košmrlj, Jan Maradin}
    \end{center}


    \section*{vprašanja za profesorja}
        \begin{itemize}
            \item ali lahko midva brez geodetskih podpoti, ker imava občutek da nama bo časa zmanjkalo
            še posebej zaradi tega da vse tako dolgo traja
            \item kakšno temperaturo morava uporabljati, sedaj imava delala predvsem 
            \item ali je okej, da midva napiševa kodo, tako da ko jo ponovno poženeva posodobi podatke,
            torej da večkrat ko poženeva kodo boljši rezultat najde, \\
            ali je poanta, da najdeva midva način, kako z dobrimi približki prideva že od enem pogonu do 
            dobrega približka
            \item kakšno analizo morava narediti, ali lahko damo v R v tabelo ali v excel?

        \end{itemize}



    \section*{Najini podvigi}

    \begin{itemize}
        \item 
        \item  ko sva probala za n = 9 in N = 4000 izvesti po Janovi metodi smo porabili
        do mu= 18  je rabil 20 minut kar je očitno čisto preveč počasi, saj bi s tem
        tempom delal za vse mu pri n = 9 rabil po vsej verjetnosti več ur
        \item Zagnala sva iskanje minimuma za $n = 9$ po Janovi metodi z nastavitvami
        \texttt{max\_steps = 300}, \texttt{T\_start = 3}, \texttt{T\_end = 0.001},
        \texttt{cooling = 0.99}. Računanje je trajalo približno 32 minut.\\
        naslednji ko sva pognala je trajalo 41 min (opomba: runnala sta dve funkcije hkrati)
        Problem: ker za $n = 9$ ne začnemo več z dobrim približkom iz $n = 8$, se zgodi,
        da se pri dodajanju ene povezave ostale povezave ne spremenijo in se vrednost
        \texttt{subpath\_number} sploh ne izboljša. Sklepava, da je težava v premajhnem
        številu korakov, da bi metoda našla boljši rezultat, torej da sva obstala v
        nekem lokalnem minimumu.

        \item Sedaj bova za $n = 9$ uporabila še metodo za manjše grafe: za vsak par
        $(n, \mu)$ poiščeva vse možne povezane grafe, za vsakega izračunava najmanjši
        \texttt{subpath\_number} in nato shraniva tistega z najmanjšo vrednostjo
        (če bodo rezultati še izvedljivi v razumnem času). \\
        v 80 minutah je funkcija, ki generira vse grafe dokončala izračune za vse grafe do vključno
        n = 9

        ogotovitev: srednje vrednosti se precej razlikujejo, ampakk ne ostalo je pa obiskuje

        \item  sedaj poganjava kodo še za max, tako da vpisuje v isti csv kot sva že imela rezulatate za min
        to delava v PRAVILNO2, trajalo je 37 min

        \item Pognala sva kodo \texttt{vecji\_grafi} za $n = 8$, da bova rezultate primerjala
        z različico \texttt{\_live}. Izvajala sva jo za minimum in maksimum ter za različno
        število ponovitev (300 in 4000). Prav tako sva v kodo dodala možnost nastavljanja
        $N$ za poljuben $n$, tako da program sam izračuna $\mu_{\text{max, prev}}$ in
        $\mu_{\text{max, n}}$. \\
        za max je trajalo 8 minut pri 4000 \\
        za max T=30 N= 300--- 3,5min \\
        za min T= 6 N= 2000 --- 

        \item Sedaj bova poskusila kodo pospešiti tako, da jo zaženeva za več vrednosti
        $\mu$ hkrati (da bolje izkoristiva 100\% CPU-ja) ter obenem povečava število
        korakov, tako da bo čas izvajanja ostal približno enak.

    \end{itemize}

        \section*{Problemi}

    \begin{itemize}
        \item ali koda ki ne zapisuje pravilno dela bolje ot tista v pravilno?
         ali je bilo takrat samo naključje
    \end{itemize}

    \section*{to do-- če bo čas}     
        \begin{itemize}
        \item poženi za n= 9 še za 2000
        \item prveri dodatne lastnosti v grafih
    \end{itemize}


    \section*{game plan}     

\begin{itemize}
    \item Za $n = 9$ uporabiva rezultate iz \texttt{ne\_live\_8}.
    \item Napiši funkcijo, ki izračuna minimum in maksimum hkrati za poljuben $n$.
    \item Izboljšaj izkoriščenost CPU-ja v tej funkciji.
    \item Zapiši funkcijo, ki bo rezultate naključnih grafov za $n = 8, 9, 10$ zapisovala v CSV.
    \item Opraviva analizo podatkov.
    \item Zapiši funkcijo, ki bo rezultate kaktusnih grafov za $n = 8, 9, 10$ zapisovala v CSV.
    \item preverit je treba če so slučajno ptc grafi pri majhnih grafih
\end{itemize}



\end{document}
