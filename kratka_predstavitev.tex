\documentclass[a4paper,12pt]{article}
\usepackage[slovene]{babel}
\usepackage[utf8]{inputenc}
\usepackage[T1]{fontenc}
\usepackage{amsmath, amssymb}
\usepackage{graphicx}
\usepackage{geometry}
\usepackage{amsthm}

\theoremstyle{definition}
\newtheorem{definition}{Definicija}

\theoremstyle{remark}
\newtheorem*{remark}{Opomba}

\geometry{margin=2.5cm}
\setlength{\parskip}{0.8em}
\setlength{\parindent}{0pt}

\begin{document}

\begin{center}
\Large\textbf{Kratko poročilo – projektna naloga}\\[4pt]
\large \textit{Skupina 18: Število podpoti pri grafih z danim ciklomatskim številom}
\end{center}

\section*{Uvod}

V projektu obravnavamo \emph{povezane} grafe $G = (V, E)$, za katere je definirano \emph{ciklomatsko število}
\[
\mu(G) = |E| - |V| + 1.
\]
To število pove, koliko neodvisnih ciklov vsebuje graf oziroma koliko povezav bi morali odstraniti, da bi dobili gozd (acikličen graf).

\textbf{CILJ NALOGE}: raziskati, kako se obnaša \emph{število podpoti in geodetskih podpoti grafa}, torej število vseh enostavnih poti (vključno s potmi dolžine 0), pri grafih z danim ciklomatičnim številom.
Zanima nas, kateri grafi imajo za dano število vozlišč $|V| = n$ in dano ciklomatsko število $ \mu = k$ najmanjše oziroma največje število poti.

\section*{Teoretično ozadje}

\begin{definition}[Podpot]
    Naj bo $G = (V, E)$ povezan graf in $P = v_1 v_2 \dots v_k$ pot v grafu $G$. 
    \emph{Podpot} grafa $G$ je vsaka zaporedna podzaporedje oglišč te poti, torej vsaka pot oblike
    \[
    v_i v_{i+1} \dots v_j, \quad \text{kjer } 1 \le i < j \le k.
    \]
    Subpath number $p_n(G)$ štejemo kot skupno število vseh enostavnih poti v grafu, pri čemer se šteje tudi vsako vozlišče kot pot dolžine 0.
\end{definition}
\smallskip

\begin{definition}[Geodetska pot in geodetska podpot]
    Naj bo $G$ povezan graf. Geodetska pot (angl.~\emph{geodesic path}) med vozliščema $u$ in $v$ 
    je pot z najmanjšim možnim številom povezav med $u$ in $v$, torej najkrajša pot med njima. 
    Geodetska podpot pa je vsaka podpot geodetske poti.
\end{definition}

\textbf{Kaktusni graf} je graf, v katerem se poljubna dva cikla stikata v največ enem vozlišču. 
To pomeni, da se cikli lahko dotikajo, vendar se ne prekrivajo.

\medskip
\textbf{Graf PTC(n,k)} (angl.~\emph{Pseudo Triangle Chain}), slovensko psevdotrikotna veriga, 
je poseben primer kaktusnega grafa, v katerem so vsi cikli trikotniki, povezani v verigo, 
in dosega največje možno število podpoti med vsemi takšnimi grafi.
Zato se v projektu pogosto uporabi kot osnovni primer (tj. zgornja meja) 
za primerjavo z drugimi grafi, ki imajo enako ciklomatično število~$\mu(G)$.

\bigskip
Graf $PTC(n,k)$ ima naslednje lastnosti:
\begin{itemize}
    \item je kaktusni graf z $n$ vozlišči in $k$ cikli;
    \item vsak notranji cikel je trikotnik (ima 3 vozlišča);
    \item cikli so med seboj povezani v verigo (en za drugim);
    \item prvi in zadnji cikel se lahko rahlo razlikujeta po velikosti (imata lahko 4 vozlišča namesto 3);
    \item vsi cikli se stikajo v natanko enem vozlišču, zato graf tvori verižasto strukturo.
\end{itemize}

\textbf{Izrek} (Knor, 2025) 
Graf $PTC(n, k)$ edinstveno maksimira število podpoti med vsemi kaktusnimi grafi z $n$ vozlišči in $k \ge 2$ cikli.

\smallskip
Ta izrek lahko torej uporabimo kot "izhodišče", kaseneje pa bomo gledali ne samo kaktusne ampak tudi splošnejše grafe.
\bigskip

\textbf{Iz literature (Knor, 2025):} za kaktusne grafe veljajo naslednji ekstremni primeri
\begin{itemize}
    \item maksimalni $p_n(G)$ doseže t.i. \emph{pseudo triangle chain}, kjer so vsi notranji cikli trikotniki, oba končna pa se razlikujeta največ za eno vozlišče;
    \item minimalni $p_n(G)$ doseže graf, kjer so vsi cikli \emph{end-trikotniki}, torej vsak trikotnik deli največ eno skupno vozlišče z ostalim grafom.
\end{itemize}


\section*{Plan dela}
Projektno nalogo bova ločila na dva dela. \\
\textbf{1. Analiza za majhne grafe.}  
V okolju SageMath bomo s pomočjo funkcije $geng$ generirali vse povezane grafe z števili vozlišč $n \le 8$ in za vsak graf izračunali:
$|V|, |E|, \mu(G) \text{ in }p_n(G).$
Pri majhnih grafih bomo število poti izračunali z iskanjem vseh enostavnih poti (DFS) med pari vozlišč.
Ker SageMath ne vsebuje vgrajene funkcije za štetje vseh enostavnih poti,
smo uporabili rekurzivni pristop. Funkcija \texttt{all\_simple\_paths(G, start, end)}
poišče vse poti med vozliščema \texttt{start} in \texttt{end}, tako da
postopno obiskuje sosednja vozlišča, pri čemer se izogiba že obiskanih točk.
Za vsak par vozlišč $(u, v)$ se nato prešteje število poti in sešteje.
Na podlagi rezultatov bomo določili grafe, ki imajo minimalni in maksimalni $p_n(G)$ pri enakem $\mu(G)$ ter preverili, ali so ti grafi kaktusnega tipa. 

\textbf{2. Eksperimentiranje za večje grafe.}  
Za $|v| > 8$ bomo uporabili naključne povezane grafe, ustvarjene z metodo \texttt{graphs.RandomGNP(n,p)}, pri čemer bomo vzdrževali število povezav $|E| = n - 1 + k$.
Za iskanje ekstremnih grafov bomo uporabili \emph{metahevristične algoritme}, kot sta \emph{simulirano ohlajanje} ali \emph{genetski algoritem}, ki bosta postopno spreminjala strukturo grafa in iskala konfiguracijo z večjim ali manjšim $p_n(G)$.
Pri tem bomo analizirali, ali optimizacija privede do struktur, podobnih znanim ekstremnim kaktusom (verige trikotnikov) ali do popolnoma novih topologij.

\section*{Pričakovani rezultati in hipoteze}
Verjamemo, da bo $p_n(G)$ najmanjši pri drevesih in večji pri grafih z večjim ciklomatskim številom.Na podlagi dane literature,
pričakujemo, da bo minimalni $p_n(G)$ pri zaporedno vezanih ciklih v kaktusnem grafu(\emph{Pseudo Triangle Chain}) in maksimalen v primeru, ko bodo vsi cikli imeli isto skupno vozlišče.
Pri večjih grafih pričakujemo, da bodo pojavljeni lokalni ekstremi, kjer več prekrivajočih se ciklov povzroči le majhno povečanje števila poti zaradi redundance.



\section*{Zaključek}

Projekt bo združil teoretično analizo in eksperimentalne rezultate ter preveril, ali so ekstremni kaktusni grafi resnično ekstremni tudi v širšem razredu grafov z enakim ciklomatskim številom.
Poseben poudarek bo na primerjavi med različnimi strukturami grafov in vplivu dodajanja povezav na število poti.
Na ta način bomo poskusili prispevati k razumevanju vedenja inverzov grafa glede na topološko kompleksnost, merjeno s ciklomatskim številom.

\end{document}
