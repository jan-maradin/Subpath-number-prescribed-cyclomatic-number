\documentclass[a4paper,12pt]{article}
\usepackage[slovene]{babel}
\usepackage[utf8]{inputenc}
\usepackage[T1]{fontenc}
\usepackage{amsmath, amssymb}
\usepackage{graphicx}
\usepackage{geometry}
\geometry{margin=2.5cm}
\setlength{\parskip}{0.8em}
\setlength{\parindent}{0pt}

\begin{document}

\begin{center}
\Large\textbf{Poročilo – projektna naloga}\\[4pt]
\large \textit{Skupina 18: Subpath number pri grafih z danim ciklomatskim številom}
\end{center}

\section*{Uvod}

V projektu obravnavamo povezane grafe $G = (V, E)$, za katere je definirano \emph{ciklomatsko število}
\[
\mu(G) = |E| - |V| + 1.
\]
To število pove, koliko neodvisnih ciklov vsebuje graf oziroma koliko povezav bi morali odstraniti, da bi dobili gozd (acikličen graf). 
Grafi z določenim $\mu(G) = k$ predstavljajo naravno posplošitev dreves ($\mu = 0$) in unocikličnih grafov ($\mu = 1$).
Ti grafi se pojavljajo v številnih strukturnih in kombinatornih problemih, npr. pri načrtovanju omrežij, električnih vezjih in analizi programske kode.

Cilj naloge je raziskati, kako se obnaša \emph{subpath number} grafa, torej število vseh enostavnih poti (vključno s potmi dolžine 0), pri grafih z danim $\mu(G) = k$.
Zanima nas, kateri grafi imajo za dano število vozlišč $n$ in dano ciklomatsko število $k$ najmanjše oziroma največje število poti.

\section*{Teoretično ozadje}

Za vsak povezan graf velja $\mu(G) = |E| - |V| + 1$. 
Pri drevesih je $\mu(G)=0$, pri dodajanju ene dodatne povezave dobimo unociklični graf ($\mu=1$), vsaka nova neodvisna povezava pa poveča $\mu(G)$ za $1$.
Subpath number $p_n(G)$ štejemo kot skupno število vseh enostavnih poti v grafu, pri čemer se šteje tudi vsako vozlišče kot pot dolžine 0.

Iz literature (Knor et al., 2025) so znani ekstremni primeri za \emph{kaktus grafe}, kjer se poljubna dva cikla stikata največ v enem vozlišču. 
Za te grafe velja:
\begin{itemize}
    \item maksimalni $p_n(G)$ doseže t.i. \emph{pseudo triangle chain}, kjer so vsi notranji cikli trikotniki, oba končna pa se razlikujeta največ za eno vozlišče;
    \item minimalni $p_n(G)$ doseže graf, kjer so vsi cikli \emph{end-trikotniki}, torej vsak trikotnik deli največ eno skupno vozlišče z ostalim grafom.
\end{itemize}
Namen projekta je preveriti, ali te lastnosti veljajo tudi pri splošnih grafih z enakim $\mu(G)$ in kako se subpath number spreminja pri različnih strukturah.

\section*{Načrt raziskave}

\textbf{Analiza za majhne grafe.}  
V okolju SageMath bomo generirali vse povezane grafe do velikosti $n \le 8$ in za vsak graf izračunali:
\[
|V|, \quad |E|, \quad \mu(G), \quad \text{in približen } p_n(G).
\]
Pri majhnih grafih bomo število poti izračunali z iskanjem vseh enostavnih poti (DFS) med pari vozlišč.
Na podlagi rezultatov bomo določili grafe, ki imajo minimalni in maksimalni $p_n(G)$ pri enakem $\mu(G)$ ter preverili, ali so ti grafi kaktusnega tipa.

\textbf{Eksperimentiranje za večje grafe.}  
Za $n > 8$ bomo uporabili naključne povezane grafe, ustvarjene z metodo \texttt{graphs.RandomGNP(n,p)}, pri čemer bomo vzdrževali število povezav $|E| = n - 1 + k$.
Za iskanje ekstremnih grafov bomo uporabili \emph{metahevristične algoritme}, kot sta \emph{simulirano ohlajanje} ali \emph{genetski algoritem}, ki bosta postopno spreminjala strukturo grafa in iskala konfiguracijo z večjim ali manjšim $p_n(G)$.
Pri tem bomo analizirali, ali optimizacija privede do struktur, podobnih znanim ekstremnim kaktusom (verige trikotnikov) ali do popolnoma novih topologij.

\section*{Pričakovani rezultati in hipoteze}

Pričakujemo, da bo minimalni graf pri danem $\mu(G)=k$ imel vse cikle skoncentrirane v enem vozlišču (analogno \emph{pseudo friendship graphu}), medtem ko bodo grafi z največjim $p_n(G)$ imeli cikle razporejene v verigo, podobno \emph{pseudo triangle chainu}.
Pri večjih grafih pričakujemo, da bodo pojavljeni lokalni ekstremi, kjer več prekrivajočih se ciklov povzroči le majhno povečanje števila poti zaradi redundance.

\section*{Plan dela}

\begin{center}
\begin{tabular}{|c|p{10cm}|}
\hline
\textbf{Dan} & \textbf{Naloga} \\
\hline
1–2 & Pregled teoretičnih osnov in članka »The subpath number of cactus graphs«. \\
\hline
3–4 & Implementacija kode za generiranje vseh povezanih grafov z $n \le 6$ in izračun $\mu(G)$. \\
\hline
5–6 & Izračun subpath numberja za majhne grafe in analiza ekstremov. \\
\hline
7–8 & Eksperimenti na večjih grafih z naključnim generiranjem in metahevristiko (simulirano ohlajanje). \\
\hline
9 & Povzetek rezultatov in primerjava s teoretičnimi pričakovanji. \\
\hline
10 & Priprava končnega poročila in priprava na zagovor. \\
\hline
\end{tabular}
\end{center}

\section*{Zaključek}

Projekt bo združil teoretično analizo in eksperimentalne rezultate ter preveril, ali so ekstremni kaktusni grafi resnično ekstremni tudi v širšem razredu grafov z enakim ciklomatskim številom.
Poseben poudarek bo na primerjavi med različnimi strukturami grafov in vplivu dodajanja povezav na število poti.
Na ta način bomo poskusili prispevati k razumevanju vedenja inverzov grafa glede na topološko kompleksnost, merjeno s ciklomatskim številom.

\end{document}
