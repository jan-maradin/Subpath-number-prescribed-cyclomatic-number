\documentclass[a4paper,12pt]{article}
\usepackage[slovene]{babel}
\usepackage[utf8]{inputenc}
\usepackage[T1]{fontenc}
\usepackage{amsmath, amssymb}
\usepackage{graphicx}
\usepackage{geometry}
\usepackage{amsthm}

\theoremstyle{definition}
\newtheorem{definition}{Definicija}

\theoremstyle{remark}
\newtheorem*{remark}{Opomba}

\geometry{margin=2.5cm}
\setlength{\parskip}{0.8em}
\setlength{\parindent}{0pt}


\begin{document}
 \textbf{Izrek} (Knor, 2025) 
        Graf $PTC(n, k)$ enolično maksimizira število podpoti med vsemi kaktusnimi grafi z $n$ vozlišči in $k \ge 2$ cikli.

        \textbf{Iz literature (Knor, 2025):} za kaktusne grafe veljajo naslednji ekstremni primeri
        \begin{itemize}
            \item maksimalni $p_n(G)$ doseže t.i. \emph{pseudo triangle chain}, kjer so vsi notranji cikli trikotniki, oba končna pa se razlikujeta največ za eno vozlišče;
            \item minimalni $p_n(G)$ doseže graf, kjer so vsi cikli \emph{end-trikotniki}, torej vsak trikotnik deli največ eno skupno vozlišče z ostalim grafom.
        \end{itemize}
\end{document}